\documentclass{article}
\usepackage[T1]{fontenc}
\usepackage[polish]{babel}
\usepackage[utf8]{inputenc}
\usepackage{lmodern}
\usepackage{multirow}
\selectlanguage{polish}
\usepackage{url}
\usepackage{graphicx} 
\title{Asseco Resovia}
\frenchspacing
\usepackage{enumerate}
\begin{document}
\maketitle    
\tableofcontents    
\newpage
                
\section{Resovia}
\label{obrazek}
\begin{figure}
\centering
\includegraphics{resovia.jpg}

\end{figure}

    Asseco Resovia SA – polski męski klub siatkarski z siedzibą w Rzeszowie. Siedmokrotny, w tym aktualny, mistrz Polski, trzykrotny zdobywca Pucharu Polski oraz zdobywca Superpucharu Polski.

Od 17 listopada 2004 klub działa jako spółka akcyjna utworzona przez AKS Resovia. Spółka zajmuje się zawodową sekcją siatkówki mężczyzn. Sponsorem tytularnym klubu jest Asseco Poland SA. Od sezonu 2006/2007 drużyna występuje w rozgrywkach pod nazwą Asseco Resovia, gdzie człon „Resovia” odnosi się do łacińskiej nazwy „Rzeszów” (podjęte uchwałą Rady Miasta Rzeszowa), a nie, jak uprzednio, do klubu CWKS Resovia, z którym umowa przestała obowiązywać.

\section{Historia}
Sekcję siatkówki założono w 1937. Momentem przełomowym dla siatkarzy był 1955, kiedy z okazji 50-lecia klubu rozpoczęto budowę hali ROSiR-u, która w późniejszych latach była świadkiem najwspanialszych sukcesów Resovii. W 1969 siatkarze grając w nowej hali, awansowali do 1 ligi.

Największe sukcesy przyszły w latach 70. „Mistrzowie podwójnej krótkiej”„”, jak popularnie nazywani byli siatkarze z Rzeszowa, największe triumfy odnieśli pod wodzą trenerów Jana Strzelczyka oraz Władysława Pałaszewskiego. Plon ich pracy to 4 złote medale mistrzostw Polski (1971, 1972, 1974, 1975), tytuł wicemistrzowski (1973), brąz mistrzostw Polski(1970) oraz udział w europejskich pucharach – II miejsce w Pucharze Europy w 1973 roku i III miejsce w Pucharze Zdobywców Pucharów w 1974 r. Największy sukces klubowe wicemistrzostwo świata w 1975. Trzon ówczesnej kadry tworzyli: Bronisław Bebel, Stanisław Gościniak, Marek Karbarz, Jan Such, Alojzy Świderek, Wiesław Radomski. Rzeszowianie odgrywali czołowe role w reprezentacji Polski oraz kadrze olimpijskiej (Bebel, Such, Gościniak i Karbarz). O sile podkarpackiego klubu świadczy fakt, że reprezentanci Polski siedzieli w Resovii na ławce rezerwowych.

W sezonie 2008/2009 Resovia po ponad 20 latach przerwy wróciła na podium mistrzostw Polski, zdobywając tytuł wicemistrzowski.
W sezonie 2009/2010 Resovia zadebiutowała w rozgrywkach Ligi Mistrzów, gdzie dotarła do 1/6 finału. Z gry w Final Four Resovię wyeliminował zespół Trentino Volley, późniejszy triumfator zawodów.
W sezonie 2010/2011 Resovia dotarła do półfinału Pucharu Cev. Z gry w finale Resovię wyeliminował zespół Sisley Treviso, późniejszy triumfator zawodów. Resovia zakończyła sezon z brązowym medalem Mistrzostw Polski. Po 3-letniej współpracy z klubem pożegnał się trener Ljubomir Travica, a jego następcą został dotychczasowy II trener - Andrzej Kowal[3].
W sezonie 2011/2012 Resovia dotarła do finału Pucharu CEV. Musiała jednak uznać wyższość zespołu Dinamo Moskwa, przegrywając w dwumeczu 0–2. Po 37 latach oczekiwania, w 2012 roku, drużyna Asseco Resovii zdobyła tytuł Mistrza Polski, pokonując w finale Skrę Bełchatów w stosunku 3–1.
W sezonie 2012/2013 Resovia po raz drugi z rzędu zdobyła tytuł Mistrza Polski, pokonując w finale ZAKSA Kędzierzyn-Koźle w stosunku 3-2.

Natomiast w sezonie 2013/2014 Resovia zdobyła srebrny medal, wygrywając w półfinale z Zaksą, ale przegrywając w finale ze Skrą Bełchatów.

W sezonie 2014/2015 Resovia po roku odzyskała mistrzostwo kraju, pokonując Trefl Gdańsk w stosunku 3-0.
\section{Bilans sezon po sezonie}

\label{tabela1}
\begin{tabular}{|c|c|c|c|c|} \hline
Sezon &	Miejsce & Mecze & Z-P & Puchary \\ \hline
 
1968/1969	&1.	& & & 	\\	\hline

1969/1970	&3.& & &\\		\hline	
1970/1971	&1. & & &\\		\hline	
1971/1972	&1. & & &\\\hline
	
1972/1973	&2.	&	&	&2. miejsce w Pucharze Europy Mistrzów Klubowych\\\hline
\multirow{2}{*}{1973/1974}	&1.	&	&	&Finalista Pucharu Polski,\\
							&	&	&	&3. miejsce w Pucharze Zdobywców Pucharów\\\hline
1974/1975	&1.	&	&	&Puchar Polski\\\hline
\multirow{2}{*}{1975/1976}	&7.	&	&	&4. miejsce w Pucharze Europy Mistrzów Klubowych,\\
						    &	&	&	&2. miejsce w KMŚ\\\hline
1976/1977	&3.	 & & &\\		\hline

1977/1978	&8.	 & & &\\		\hline
1978/1979	&8.	 & & &\\		\hline
1979/1980	&4.	 & & &\\	\hline	
1980/1981	&5.	 & & &\\	\hline	
1981/1982	&4.	 & & &\\	\hline	
1982/1983	&8.	&&&		Puchar Polski\\\hline
1983/1984	&6.	 & & &\\		\hline
1984/1985	&4.	 & & &\\		\hline
1985/1986	&6.	&	&	&Finalista Pucharu Polski\\\hline
1986/1987	&3.	&	&	&Puchar Polski\\\hline
1987/1988	&3.	&&&		\\\hline
1988/1989	&6.	&	&	&3. miejsce w Pucharze Polski\\\hline
1989/1990	&8.	 & & &\\		\hline
1990/1991	&8.	 & & &\\		\hline
1991/1992	& & & &\\		\hline
1992/1993	& & & &\\			\hline
1993/1994	& & & &\\			\hline
1994/1995	& & & &\\			\hline
1995/1996	& & & &\\			\hline
1996/1997	& & & &\\			\hline
1997/1998	& & & &\\			\hline
1998/1999	& & & &\\			\hline
1999/2000	& & & &\\			\hline
2000/2001	& & & &\\			\hline
2001/2002	& & & &\\			\hline
2002/2003		& & & &\\		\hline
2003/2004	&1.	&28	&20-8&	\\ \hline
2004/2005	&7.	&26	&9-17&	\\ \hline
2005/2006	&7.	&28	&12-16&	\\ \hline
2006/2007	&5.	&27	&13-14&	\\ \hline
2007/2008	&5.	&29	&16-13	&4. miejsce w Pucharze Challenge\\ \hline
2008/2009	&2.	&28	&19-9	&\\ \hline
2009/2010	&3.	&31	&19-12	&Finalista Pucharu Polski\\ \hline
2010/2011	&3.	&35	&24-11	&\\ \hline
2011/2012	&1.	&29	&22-7	&2. miejsce w Pucharze CEV\\ \hline
2012/2013	&1.	&32	&21-11	&Finalista Pucharu Polski\\ \hline
2013/2014	&2.	&33	&24-9	&\\ \hline
\multirow{2}{*}{2014/2015}	&1.	&34	&30-4	&2. miejsce w Lidze Mistrzów, \\
							&	&	&	&Finalista Pucharu Polski\\ 
\hline	
\end{tabular}	


\subsection{Sukcesy}
\begin{enumerate}
\item Mistrzostwo Polski:
\begin{itemize}
\item 1. miejsce (7x): 1971, 1972, 1974, 1975, 2012, 2013, 2015
\item 2. miejsce (3x): 1973, 2009, 2014
\item 3. miejsce (6x): 1970, 1977, 1987, 1988, 2010, 2011
\end{itemize}
\item Puchar Polski:
\begin{itemize}
\item 1. miejsce (3x): 1975, 1983, 1987
\item 2. miejsce (5x): 1974, 1986, 2010, 2013, 2015
\item 3. miejsce (1x): 1989
\end{itemize}
\item Superpuchar Polski:
\begin{itemize}
\item 1. miejsce (1x): 2013
\item 2. miejsce (1x): 2012, 2015
\end{itemize}
\item Puchar Europy Mistrzów Klubowych / Liga Mistrzów:
\begin{itemize}
\item 2. miejsce (2x): 1973, 2015
\item 4. miejsce (1x): 1976
\end{itemize}
\item Puchar Zdobywców Pucharów / Puchar CEV:
\begin{itemize}
\item 2. miejsce (1x): 2012
\item 3. miejsce (1x): 1974, 2011
\item 4. miejsce (1x): 1987
\end{itemize}
\item Puchar Challenge:
\begin{itemize}
\item 4. miejsce (1x): 2008
\end{itemize}
\end{enumerate}
\section{Kadra}
\begin{itemize}
\item Pierwszy trener: Polska Andrzej Kowal
\item Drugi trener: Polska Marcin Ogonowski
\item Pierwszy statystyk: Polska Michał Mieszko Gogol
\item Drugi statystyk: Polska Sergiusz Ruszel
\item Trener przygotowania fizycznego: Polska Łukasz Filipecki
\item Fizjoterapeuta: Polska Jacek Rusin
\item II Fizjoterapeuta: Polska Mateusz Przystaś
\item Kierownik drużyny: Polska Wojciech Groszek
\item Menedżer drużyny: Polska Maciej Pająk
\end{itemize}
\begin{tabular}{|l|l|c|c|c|} \hline
Nr	&Imię i nazwisko	&Data ur.	&Wzrost	&Pozycja \\ \hline 
1	&Bartosz Kurek	&29.08.1988	&205	&atakujący\\ \hline 
2	&Thomas Jaeschke	&04.09.1993	&200	&przyjmujący\\ \hline 
3	&Kowalski	&20.02.1997	&200	&środkowy\\ \hline 
4	&Piotr Nowakowski	&18.12.1987	&205	&środkowy\\ \hline 
5	&Lukas Tichacek	&12.01.1982	&193	&rozgrywający\\ \hline 
6	&Dawid Dryja	&21.07.1992	&201	&środkowy\\ \hline 
7	&Olieg Achrem	&12.03.1983	&195	&przyjmujący\\ \hline 
8	&Julien Lyneel	&15.04.1990	&192	&przyjmujący\\ \hline 
9	&Dmytro Paszycki	&29.11.1987	&202	&środkowy\\ \hline 
10	&Jochen Schöps	&08.10.1983	&204	&atakujący\\ \hline 
11	&Fabian Drzyzga	&03.01.1990	&196	&rozgrywający\\ \hline 
12	&Łukasz Perłowski	&03.04.1984	&203	&środkowy\\ \hline 
13	&Marcin Karakuła	&23.12.1998	&190	&rozgrywający\\ \hline 
14	&Aleksander Śliwka	&24.05.1995	&202	&przyjmujący\\ \hline 
15	&Michał Marszałek	&17.04.1998	&191	&przyjmujący\\ \hline 
16	&Krzysztof Ignaczak	&15.05.1978	&188	&libero\\ \hline 
17	&Nikołaj Penczew	&22.05.1992	&196	&przyjmujący\\ \hline 
18	&Damian Wojtaszek	&07.09.1988	&180	&libero\\ \hline 
19	& Russell Holmes	&01.07.1982	&205	&środkowy\\ \hline 
20	&Dominik Witczak	&02.01.1983	&198	&atakujący\\ \hline 
&Bartłomiej Lemański (na wyp.)	&19.03.1996	&215	&środkowy\\ \hline 
&Michał Filip (na wyp.)	&31.08.1994&	195&	przyjmujący\\ \hline 
&Dominik Depowski (na wyp.)	&27.10.1995	&203	&przyjmujący\\ \hline 
&Bartłomiej Mordyl (na wyp.)	&21.01.1995&	200	&środkowy\\ \hline 
&Michał Kędzierski (na wyp.)	&09.08.1994	&194	&rozgrywający\\ \hline 
&Michał Kozłowski (na wyp.)	&16.02.1985	&197	&rozgrywający\\ \hline 
&Łukasz Kozub	&03.11.1997	&188	&rozgrywający\\ 
\hline	
\end{tabular}	
\subsection{Trenerzy}
\begin{tabular}{|l|c|} \hline
Nazwisko	&Okres\\ \hline
Jan Such	&2003-3 grudnia 2007\\ \hline
Andrzej Kowal	&3 grudnia 2007-2008\\ \hline
Ljubomir Travica	&2008-20 kwietnia 2011\\ \hline
Kowal	&20 kwietnia 2011-dziś\\ \hline
\end{tabular}
\subsection{Obcokrajowcy w zespole}
\begin{tabular}{|c|c|c|} \hline
Lata Gry	&Imię i nazwisko	&Pozycja\\ \hline
2004-2005	&Siergiej Żywołożnyj	&przyjmujący\\ \hline
2005-2007	&Dušan Kubica	&libero\\ \hline
2006-2008	&Karel Kvasnička	&przyjmujący\\ \hline
2006-2007	&Geir Eithun	&przyjmujący\\ \hline
2007-2011-	&Ivan Ilić	&rozgrywający\\ \hline
2007-2009	&Ihosvany Hernández Rivera	&środkowy\\ \hline
2007-2009	&Aleksander Mitrović	&przyjmujący\\ \hline
2008-2010	& Mikko Oivanen	&atakujący\\ \hline
2009-	&Olieg Achrem	&przyjmujący\\ \hline
2009-2009	& Ardo Kreek	&środkowy\\ \hline
2010-2012	& Georg Grozer	&atakujący\\ \hline
2010-2011	& Michele Baranowicz	&rozgrywający\\ \hline
2010-2011	& Matej Černič	&przyjmujący\\ \hline
2010-2011	&Ryan Millar	&środkowy\\ \hline
2011-2015	&Paul Lotman	&przyjmujący\\ \hline
2011-	& Lukáš Ticháček	&rozgrywający\\ \hline
2011-2012	& Adrian Radu Gontariu	&przyjmujący\\ \hline
2011-2012	& Marko Bojić	&przyjmujący\\ \hline
2012-	& Jochen Schöps	&atakujący\\ \hline
2012-2013	& Nikola Kovačević	&przyjmujący\\ \hline
2013-	& Nikołaj Penczew	&przyjmujący\\ \hline
2013-2014	& Péter Veres	&przyjmujący\\ \hline
2014-2015	&Marko Ivović	&przyjmujący\\ \hline
2014-	&Russell Holmes	&środkowy\\ \hline
2015-	&Dmytro Paszycki	&środkowy\\ \hline
2015-	&Thomas Jaeschke	&przyjmujący\\ \hline
2015-	&Julien Lyneel	&przyjmujący\\ \hline
\end{tabular}
\section{Klub kibica}
Stowarzyszenie nosi nazwę Stowarzyszenie Sympatyków Piłki Siatkowej Resovia w Rzeszowie (w skrócie SSPS Resovia), którego siedzibą jest miasto Rzeszów. Działa na podstawie ustawy Prawo o stowarzyszeniach (Dz.U.nr.20 z 1989 r. Poz. 104 z późniejszymi zmianami).

\section{Sekcja na potrzeby zaliczenia}
Odsyłacz do obrazka: \ref{obrazek} \\
Tabela: \ref{tabela1} \\
\label{wzor}
Wzór matematyczny:
$\sum_{i=1}^k a_{i} = a_{1} + a_{2} + a_{3} + ... + a_{k}$ \\
Odsyłacz do wzoru: \ref{wzor}



\cleardoublepage % 
\renewcommand*{\refname}{}
\section{Bibliografia} 
\begin{thebibliography}{9}

\bibitem{bl}Wikipedia, \url{https://pl.wikipedia.org/wiki/Resovia_(pi%C5%82ka_siatkowa)}, 23 12 2011.
\bibitem{kmecner}
  Krzysztof Mecner,
  80 lat polskiej siatkówki.
  Olsztyn: Hakus,
  2001,
  ISBN 83-86926-08-2.
 \bibitem{mkalita}
  Marcin Kalita,
  Polska siatkówka.
  Ringier Axel Springer,
  Polska,
 ISBN 978-83-7813-410-7.
\end{thebibliography}
\end{document}
